%
% Installation
%
\begin{frame}[fragile]
  \frametitle{Installation}
  \begin{itemize}
  \item install packages
    \lstset{style=mybash}
    \begin{lstlisting}
zypper ar -f obs://devel:openQA/openSUSE_13.1 openQA
zypper ar -f obs://devel:openQA:13.1/openSUSE_13.1 \
  openQA-perl-modules
zypper in openQA apache2
    \end{lstlisting}
    \item start web interface
    \begin{lstlisting}
systemctl start openqa-webui
    \end{lstlisting}
  \item More detailed instructions at \newline
    \url{https://github.com/os-autoinst/openQA}
  \end{itemize}
\end{frame}

\begin{frame}[fragile]
  \frametitle{Apache Setup}
  \begin{itemize}
    \item use default vhost template
    \lstset{style=mybash}
    \begin{lstlisting}
cp /etc/apache2/vhosts.d/openqa.conf.template \
   /etc/apache2/vhosts.d/openqa.conf
    \end{lstlisting}
\item enable required apache modules
    \begin{lstlisting}
a2enmod headers
a2enmod proxy
a2enmod proxy_http
    \end{lstlisting}
    \item (re)start apache
    \begin{lstlisting}
rcapache2 restart
    \end{lstlisting}
  \end{itemize}
\end{frame}

\begin{frame}[fragile]
  \frametitle{Generate Secrets for Authentication}
  \begin{itemize}
    \item switch off https in \texttt{/etc/openqa/openqa.ini}
    \begin{lstlisting}
[openid]
httpsonly = 0
    \end{lstlisting}
    \item go to \url{http://localhost/} and log in
    \item go to Admin -> Secrets and generate a pair of key+secret
    \item edit \texttt{/etc/openqa/client.conf} and put key and secret there
  \end{itemize}
\end{frame}

% opensuse specific, we have dsl as example later though
%\begin{frame}[fragile]
%  \frametitle{fetch tests and reference images}
%  \begin{itemize}
%    \item the \texttt{fetchneedles} tool gets openSUSE data from github
%    \lstset{style=mybash}
%    \begin{lstlisting}
%/usr/lib/os-autoinst/tools/fetchneedles
%    \end{lstlisting}
%  \end{itemize}
%\end{frame}
